\documentclass{article}
\usepackage{amsmath}
\usepackage{amssymb}
\usepackage{cancel}
\usepackage{setspace}
\usepackage{graphicx}
\usepackage{enumitem}
\usepackage[colorlinks=true, allcolors=blue]{hyperref}
\usepackage[english]{babel}
\usepackage[letterpaper,top=2cm,bottom=2cm,left=1cm,right=2cm,marginparwidth=1 cm]{geometry}
\usepackage{nicefrac, xfrac}
\usepackage{indentfirst}
\onehalfspacing
\begin{document}
\setlength\parindent{0pt}
\section{Sayisal Turev icin Ileri Yon Sonlu Fark Formulleri}
\subsection{Taylor Serisi}
Bir $f(x)$ fonksiyonunun $x=x_{0}$ noktasindaki Taylor serisine acilimi asagidaki ifade ile hesaplanir.\\
$\displaystyle f(x)=\sum_{n=0}^{\infty}\frac{f^{(n)}(x_{0})}{n!}(x-x_{0})^n = f(x_{0}) + \frac{f'(x_{0})}{1!}(x-x_{0}) + \frac{f''(x_{0})}{2!}(x-x_{0})^2 + \frac{f'''(x_{0})}{3!}(x-x_{0})^3 + \cdot\cdot\cdot$
\subsection{Ornek Noktalar}
Bu ifade $x_{m}=x_{0}+mh$ biciminde secilen her bir nokta icin duzenlenir.\\
$\displaystyle y_{-1}=f(x_{0})+(-1)*h*f'(x_{0})+(-1)^{2}*\frac{h^{2}}{2!}*f''(x_{0})+(-1)^{3}*\frac{h^{3}}{3!}*f'''(x_{0})$\\
$\displaystyle y_{1}=f(x_{0})+h*f'(x_{0})+\frac{h^{2}}{2!}*f''(x_{0})+\frac{h^{3}}{3!}*f'''(x_{0})$\\
$\displaystyle y_{2}=f(x_{0})+2*h*f'(x_{0})+2^{2}*\frac{h^{2}}{2!}*f''(x_{0})+2^{3}*\frac{h^{3}}{3!}*f'''(x_{0})$\\
\subsection{Katsayilar ile carpim}
Denklemler $t_{1}, t_{2}, t_{3}, ...$ katsayilari ile carpilir.\\
$\displaystyle t_{1}*y_{-1}=t_{1}*f(x_{0})+t_{1}*(-1)*h*f'(x_{0})+t_{1}*(-1)^{2}*\frac{h^{2}}{2!}*f''(x_{0})+t_{1}*(-1)^{3}*\frac{h^{3}}{3!}*f'''(x_{0})$\\
$\displaystyle t_{2}*y_{1}=t_{2}*f(x_{0})+t_{2}*h*f'(x_{0})+t_{2}*\frac{h^{2}}{2!}*f''(x_{0})+t_{2}*\frac{h^{3}}{3!}*f'''(x_{0})$\\
$\displaystyle t_{3}*y_{2}=t_{3}*f(x_{0})+t_{3}*2*h*f'(x_{0})+t_{3}*2^{2}*\frac{h^{2}}{2!}*f''(x_{0})+t_{3}*2^{3}*\frac{h^{3}}{3!}*f'''(x_{0})$\\
\subsection{Birlestirme}
Denklemlerin her iki yani toplanir.\\
$\displaystyle t_{1}*y_{-1}+t_{2}*y_{1}+t_{3}*y_{2}=(t_{1}+t_{2}+t_{3})*f(x_{0})+(t_{1}*(-1)+t_{2}+t_{3}*2)*h*f'(x_{0})+(t_{1}*(-1)^{2}+t_{2}+t_{3}*2^{2})*\frac{h^{2}}{2!}*f''(x_{0})+(t_{1}*(-1)^{3}+t_{2}+t_{3}*2^{3})*\frac{h^{3}}{3!}*f'''(x_{0})$
\subsection{Katsayi ifadeleri}
Bu toplam ifadesinden her bir terimin katsayi ifadesi cikarilir.\\
$\displaystyle Exp_{0}: t_{1}+t_{2}+t_{3}$\\
$\displaystyle Exp_{1}: t_{1}*(-1)+t_{2}+t_{3}*2$\\
$\displaystyle Exp_{2}: t_{1}*(-1)^{2}+t_{2}+t_{3}*2^{2}$\\
$\displaystyle Exp_{3}: t_{1}*(-1)^{3}+t_{2}+t_{3}*2^{3}$\\
\subsection{Denklem Sistemi}
Bu katsayi ifadelerinden ilki ile 3. turevin ifadesi cikarilarak asagidaki denklemler elde edilir.\\
$\displaystyle t_{1}*(-1)+t_{2}+t_{3}*2=0$\\
$\displaystyle t_{1}*(-1)^{2}+t_{2}+t_{3}*2^{2}=0$\\
Bu denklemlerden katsayi matrisi olusturulur.
\begin{center}
$$ \left[\begin{array}{rrr}
t_{1} & t_{2} & t_{3}\\
(-1) & 1 & 2\\
(-1)^{2} & 1 & 2^{2}\\
\end{array}\right] $$
\end{center}
Denklemler, $t_{1}$ bagimsiz degisken yapilarak (sag tarafa tasinarak) asagidaki gibi yeniden duzenlenir.\\
\begin{center}
$$ \left[\begin{array}{rr|r}
t_{1} & t_{2} & t_{3}\\
-1 & 1 & -2\\
1 & 1 & -4\\
\end{array}\right] $$
\end{center}
\subsection{Denklem cozumu}
1. satir kullanilarak  asagisindaki (2,1) elemani 0 yapilir.\begin{center}
$$ \left[\begin{array}{rr|r}
t_{1} & t_{2} & t_{3}\\
-1 & 1 & -2\\
0 & 2 & -6\\
\end{array}\right] $$
\end{center}
2. satir kullanilarak  yukarisindaki (1,2) elemani 0 yapilir.\begin{center}
$$ \left[\begin{array}{rr|r}
t_{1} & t_{2} & t_{3}\\
-1 & 0 & 1\\
0 & 2 & -6\\
\end{array}\right] $$
\end{center}
Katsayi matrisi birim matrise donusturulur.
\begin{center}
$$ \left[\begin{array}{rr|r}
t_{1} & t_{2} & t_{3}\\
1 & 0 & -1\\
0 & 1 & -3\\
\end{array}\right] $$
\end{center}
Buradan katsayi cozumleri asagidaki gibi belirlenir.\\
$\displaystyle t_{1}=-1*t_{3}$\\
$\displaystyle t_{2}=-3*t_{3}$\\
\subsection{Katsayilar icin Tamsayi Degerleri}
$t_{3}$ degiskenine atanabilecek en kucuk tamsayi degeri EKOK hesabi ile belirlenir.\\
$t_{3}=EKOK()=1$\\
\\
Bu deger yardimiyla katsayi degiskenlerine atanacak degerler hesaplanir.\\
$\displaystyle t_{1}=-1*1=-1$\\
$\displaystyle t_{2}=-3*1=-3$\\
$\displaystyle t_{3}=1$\\
\\
$y_{0}$ noktasinin katsayisi ($t_{0}$) ile birlikte tum katsayilar asagidaki gibi belirlenir.\\
$y_{0}: t_{0}=t_{1}+t_{2}+t_{3}=(-1)+(-3)+1=-3$\\
$y_{1}: t_{1}=-1$\\
$y_{2}: t_{2}=-3$\\
$y_{3}: t_{3}=1$\\
$y^{(3)}: 1*h^3$\\
\\
$y^{(3)}$ ifadesi hesaplanir.\\
$y^{(3)}=\frac{-y_{-1}+3*y_{0}-3*y_{1}+y_{2}}{1*h^{3}}+O(h)$\\
$y^{(3)}=\frac{-0.45399049973954675+3*0.5877852522924731-3*0.7071067811865475+0.8090169943749475}{1*0.15707963267948966^{3}}+O(h)$\\
$y^{(3)}=-0.7580638145+O(h)$\\
\end{document}
